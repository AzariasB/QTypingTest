%------------------------------------
% Début Partie Pierre
%------------------------------------
\part{Introduction}

\chapter{System prototype report}

\section{Abstract}
This project report will introduce you the group's third year major group project. It will present all parts of the project, from the beginning with the research of an idea to the final product through all steps. The final result is a computer's open source software called “QTypingTest“ usable on any platform (Window, Linux or Mac OS). The software is composed of different parts where users can learn how to type on a keyboard, improve what they learned and use all this in a “competition” part. Users have access to statistics to show the evolution of their learning. The software has been developed using Qt, a cross-platform application software framework which allow to easily develop software that can be run on various platforms. 

\section{Introduction}
The application developed is a learning and testing software which gives users the opportunity to learn how to type well on a keyboard, to practice and to improve. When the software runs, the user is on the home page where he can create a new account and select or delete an existing one. Then, when an account is selected or created, the user has access to all different part of the software. Those part are “Home” which send the user back to the home page, “Learn” which send the user to the section where he can learn how type and see which letters are learned and which one need to be, “Practice” when the user has access to four different practice exercises, “Statistics” where he can see statistics about his different practice and “Game” where he can find an interactive keyboard. 

\section{Brief walkthrough}
Basically, it is very simple to run the application. Indeed, the user just need the executable (.exe) on his computer. When the software is running, the user can see the menu where he can select an existing user or create a new one and then navigate on the software. 

\section{Technology}
\begin{itemize}
    \item C++ is the language for this software.
    \item Qt (pronounced ‘Cute’ or ‘Q-T’) is a framework use to develop the software. It has been choose because it is very suitable with C++ and to develop cross-platform software. The user does not need Qt on is computer to run the application. 
    \item Any software development platform (such as NetBeans or Eclipse …) could be used to code as long it supports Qt software. Qt company provides its own IDE, Qt Creator, which is very adapted because it include some tutorials and the full library documentation of the framework. 
\end{itemize}

\section{Difficulties Faced}
The challenges faced were different related to each member of the group. The principal one was at the beginning for both of the group member and was to understand how Qt was working because it was new for them. Pierre, also has some difficulties with C++ because he was less familiar with this language than Azarias. 

\section{Current Functionality}
\begin{itemize}
\item Can create a new user, use or delete an old one.
\item Users can use the "Learn" section and see the evolution of his learning.
\item Practice against time works.
\item Normal practice works.
\item Improve practice works.
\item Text practice works.
\item Users can see their own statistics.
\item Interactive keyboard.
\end{itemize}

\section{Next Stage Features}
\begin{itemize}
\item Traduce the software in other languages
\item Develop a game
\item Allow competitions between users
\end{itemize}

\section{Conclusion}
The aim of this report is to cover all steps and to present all functionalities and choices the group made. This project has in purpose to produce a simple but robust and complete computer software. The final product, as presented then, is fully working and totally match with the group beginning ideas.  

\chapter{Computing Domain}
As this project has in purpose to develop a software it was logic to use an adapted programming language for this. As the group members did C and Java in the first semester, they decide to use an other one in a way to learn or to improve our skills on an other language. That is why the group chose C++.
\\
C++ is a general object-oriented programming language who provide facilities with low-level memory manipulation. It was well adapted for our project because as C++ is a language who need to be compiled, an executable is created and thereby can be used on user's computer to run the software. 
\\
There are different frameworks (libraries) available with C++ like Gtkmm, SFML or Qt for example. The group decided to use Qt because it was new for them and thus the project would give them a way to learn how to use it. Furthermore, Qt is well adapted for this type of software and the Qt company provides a good IDE (Qt Creator), which is fully working with the Qt library. This IDE provides a GUI builder and a documentation of the library is directly included inside the IDE. 
\\
To build the design of the window's software, as said before, the group used the GUI builder provided buy Qt Creator.

\chapter{The project group}
The group is composed of two students. As they both know that a gap of level was existing and to be able to get a good final product, the group decide to try to find a way to split the work as equitably as possible keeping in mind their objectives and personal skills.
\\
Azarias was working faster and get more competence in C++ so he did more coding than Pierre. 
\\
C++ was newest for Pierre so he had to adapt and get more time to learn basics but he finally provided code. 
\\
The report writing skills was pretty equals but as Pierre coded less he tried to spend more time as Azarias on this. 

\chapter{Project deliverables}
The deliverables as specified in the project handbook are:
\begin{itemize}
\item A working prototype.
\item An oral presentation and defence of the project.
\item System prototype documentation.
\item Full software.
\item A detailed project report.
\end{itemize}

\chapter{Document outline}
This report documentation presents the entire project from the beginning with the research to the final software.
All areas provided in this report are:
\begin{itemize}
\item An introduction including a chapter containing a system prototype report, a chapter about the computing domain, a chapter about the project group, a chapter about the project deliverables and another with this document outline.
\item A complete literature review which cover the main research for the project. The different areas of research covered by this literature review are some historic researches related to the subject, different keyboard available on the market and typing ways.  
\item A system analysis part containing a chapter covering different aspect of the functional requirements. 
\end{itemize}

\part{Literature review}
%\documentclass[12pt]{report}%??????autres?choix?:?book,?report
\usepackage[utf8]{inputenc}%???????????gestion?des?accents?(source)
\usepackage[T1]{fontenc}%??????????????gestion?des?accents?(PDF)
\usepackage[english]{babel}%english gestion
\usepackage{hyperref}
\usepackage[bottom]{footmisc}
\usepackage[font=small,labelfont=bf]{caption}
\usepackage[newparttoc]{titlesec}
\usepackage[nonumberlist,toc]{glossaries}
\usepackage[export]{adjustbox}
\usepackage[margin=0.95in]{geometry}
\usepackage[square,sort,comma,numbers]{natbib}

%All the packages
\usepackage{lmodern,url,ragged2e,textcomp,lmodern,paralist}
\usepackage{graphicx,xcolor,float,afterpage}
\usepackage{chngcntr,csquotes,helvet,lastpage}
\usepackage{subcaption,wrapfig,fancyhdr,blindtext}
\usepackage{titletoc,transparent,datatool}

%List de mot ordonné
\newcommand{\sortitem}[1]{%
  \DTLnewrow{list}% Create a new entry
  \DTLnewdbentry{list}{description}{#1}% Add entry as description
}
\newenvironment{sortedlist}{%
  \DTLifdbexists{list}{\DTLcleardb{list}}{\DTLnewdb{list}}% Create new/discard old list
}{%
  \DTLsort{description}{list}% Sort list
  \begin{center}
	  \begin{inparaitem}%
    		\DTLforeach*{list}{\theDesc=description}{%
         \item \theDesc \hspace{0.1cm} }% Print each item
      \end{inparaitem}•%  
  \end{center}
}

%Nouvelles commandes
\renewcommand{\familydefault}{\sfdefault} %default font
\newcommand{\HRule}{\rule{\linewidth}{0.5mm}}
\newcommand{\Mline}{\hrule \mbox{}\\[0.1cm]}
\renewcommand{\thechapter}{\Roman{chapter}}

%final last page
\newcommand\blankpage{%
    \null
    \thispagestyle{empty}%
    \addtocounter{page}{-1}%
    \newpage}


%FORMAT DU CHAPITRE
\titleclass{\chapter}{straight}
\titleformat{\chapter}[hang]
  {}
  {\normalfont \sffamily \bfseries \thechapter.}
  {0pt}
  {\normalfont \sffamily \bfseries}
\titlespacing*{\chapter}{0pt}{50pt}{18pt}

%FORMAT De section
\renewcommand*\thesection{\arabic{section}}
\titleclass{\section}{straight}
\titleformat{\section}[hang]
  {}
  {\small \sffamily \bfseries \textit \thesection . }
  {0pt}
  {\small \bfseries \textit}
  
\titlespacing*{\chapter}{0pt}{50pt}{18pt}

%Comptage des figures
\renewcommand{\thefigure}{\arabic{figure}}

%Ne pas reset le numéro des figures  à chaque chapitre
\counterwithout{figure}{chapter}

%Custom footer and header
\pagestyle{fancy}
\fancyhf{}
%\lhead{Implement a Typing Speed Test With Qt}
\rhead{B00092351 Azarias Boutin, B00092354 Pierre Thubé}
\rfoot{Page \thepage}

\begin{document}

\begin{titlepage}
\begin{center}


% Title
\Mline
{ \LARGE Implement a Typing Speed Test With Qt \\[0.4cm] }
{ \LARGE Literature Review\\[0.4cm] }
\Mline
% Author and supervisor

\textsf{}\\[3cm]

\textsf{Boutin Azarias B00092351, B00092354 Pierre Thubé - Group 2\\[2cm]
BN013 BSC in Computing}


\end{center}
\end{titlepage}
\clearpage

\tableofcontents
\listoffigures
\chapter*{Keywords}
\begin{sortedlist}
  \sortitem{Keyboard}
  \sortitem{Keyboard layout}
  \sortitem{Fingers}
  \sortitem{Keys}
  \sortitem{Speed}
  \sortitem{Accuracy}
  \sortitem{Typing}
  \sortitem{Improvements}
  \sortitem{Learning}
\end{sortedlist}

\thispagestyle{empty}
\clearpage

\setcounter{page}{1}
\chapter{Abstract}
This review explores the way of improving the typing speed on the common keyboard.
Today's world is already full of keyboards, and it will never cease to increase. To be more efficient when using a computer, it is a good idea to learn how to type. On the web there are plenty of sources to improve your skills, but there is very little of them who really teach you how to type faster from scratch. The same goes for the existing software. And the main issue regarding this is that the software is not free.\\
The work presented here is based on books and websites that give advice on how to type speedily.\\
In order to address that issue, this review looks at the finger positions. Then it goes on to discuss how the user must practice to improve his or her typing skills. Lastly, it looks at the importance of accuracy and how this outweighs speed.

\clearpage
\chapter{Literature review}
\section{Some history}
In 1890, Lovisa Ellen Barnes wrote a book called "How To Become Expert in Typewriting"\cite{ref2} who was explaining how type on a Remington Typewriter.\\
In 1915, Dr. Henry Faulds wrote a book called "A Manual of Practical Dactylography"\cite{ref3}\\
An other example of book talking about dactylography.\cite{ref5}\\   
A more recent one published in 2012 and wrote by Faulds H.\cite{ref4}\\
As we can see, from a long time people provide way to learn how type so we can thereby say that way to type, and so typing speed, are always being important and are never been neglect by people.\\

\section{Keyboard}
The original layouts for the first mechanical typewriters were in alphabetical order (ABCDE etc..). Then different type of keyboard appeared according to languages used in each country (the most used is the QWERTY one but you can find AZERTY in France or QWERTZ in Germany for example).\\
Each of this layouts keyboard are classics one, that is mean when you buy a keyboard you will have one of this, but, since the invention of computers and indeed keyboards, a lots of research 	conducted to the invention of new non-classics layouts. Indeed, this new layouts had to made typing more easy and more comfortable. The most famous are Dvorak, Colemak and Workman.\\
Dvorak has been invented by August Dvorak and is the best alternative to QWERTY. It has been designed to increase typing speed, decrease errors and increase comfort.\\
After Dvorak, Colemak is the most popular. Also based on the QWERTY keyboard, it has been created in 2006 by Shai Coleman. It allow to promote the bearing of fingers on the middle rows of keys, it is better to maximise alternation between use right and left hands and minimize the movement of hands on the keyboard.\\
The third one is Workman. Thise one is less used than Dvorak and Colemak, but according with different developers, it is actually the best one to reduce fingers exhaustion. Indeed, it reduce usage of the two middle columns, reduce horizontal,diagonal and vertical finger stretching compared to Dvorak and Colemak.\cite{ref6}\\
To conclude about that, we can obviously say that from people have always trying to increase their typing skills by inventing new keyboards layouts which increase typing comfort and thus typing speed.     
\begin{figure}[h!]
   \begin{minipage}[b]{0.32\linewidth}
      \centering \includegraphics[scale=0.16]{images/KB_United_States_Dvorak.png}
      \caption{\it Dvorak Keyboard}
   \end{minipage}
   \begin{minipage}[b]{0.32\linewidth}   
      \centering \includegraphics[scale=0.16]{images/KB_US-Colemak_with_AltGr.png}
      \caption{\it Colemak Keyboard}
   \end{minipage}
   \begin{minipage}[b]{0.32\linewidth}
      \centering \includegraphics[scale=0.16]{images/KB_English_Workman.png}
      \caption{\it Workman Keyboard}
   \end{minipage}
\end{figure}
\clearpage

\section{Typing way and Fingers position}
To improve typing speed, in addition to change our keyboard, the only way is to practice. Actually, the average typing speed, calculate in Word Per Minute (WPM), is of 44 words.\cite{ref7} In the English language, the fastest typist in the world was Stella Pajunas who had a top speed of 216 words per minute. She realize it on a typewriter in 1946. Nowadays, it is easy with practice to had a typing speed of about 100 words par minute.\\ 
The typing way, which can be learn and improve on a lots of software and websites, is really the most important things to take in consideration to improve our typing speed. This typing way include the position of hands on a keyboard, which will change according to the type of keyboard, but for a QWERTY one it is like the figure 4 shows.
\begin{figure}[H]
\begin{center}
\includegraphics[width=9cm]{images/FingerHandPosUSA.png} 
\end{center}
\caption{\it Fingers position on QWERTY keyboard}
\label{Poulpy est multicolore}
\end{figure}
Here you can find some example of typing test on the internet:
\begin{itemize}
\item\it TypingTest.com\cite{ref8}
\item\it 10fastfingers.com\cite{ref9}
\end{itemize}
It also possible to find typing lessons on the internet.\\
To conclude about this part we can say that way to type and typing speed are things very democratize on the internet especially with a lots of website who can explain you best way to type, position of hands, give you advice and way to improve your typing style with speed test. That is showing that the development of a typing test is something useful.
Every book and website available regarding improving the typing speed are first mentions the finger positions.\\
As the book written by \citet{beginners}, the fingers must be positioned on the \textit{baseline}. The baseline, is the line on the keyboard containing the letters \textit{F} and \textit{J}. As with 'touch typing in ten lessons' \cite{tenlessons} book explains, every fingers should have an assigned letter. The figure below \ref{fing_pos} shows where the finger position is assigned on a QWERTY keyboard.
\begin{figure}[H]
	\centering
	\includegraphics[width=0.7\textwidth]{images/fing_pos.png}
	\caption{Position of the fingers on a QWERTY keyboard}\label{fing_pos}
\end{figure}
However, the layout of the keyboard may vary but the finger position will stay on the same baseline. www.computerhope.com \cite{computerhope} shows the different keys for each finger depending on the keyboard layout.
Other keyboards exist such as those represented in 'typing for beginners' \citep{beginners}. Unfortunately there are not common enough and are only used by more professional typists. So this literature review will not cover the use of them.\\
The user must know which finger is assigned to its specific letter on the keyboard. And so forth for each row. The book written by
\citet{beginners} explains the differents steps on how to follow the assignment for each row. Once again, the letters may vary depending on the keyboard layout, but the finger movements are still the same.
\begin{figure}[H]
    \centering
    \begin{subfigure}[b]{0.3\textwidth}
        \includegraphics[width=\textwidth]{images/qwerty.png}
        \caption{Qwerty-based layout}
    \end{subfigure}
	 \begin{subfigure}[b]{0.3\textwidth}
        \includegraphics[width=\textwidth]{images/azerty.png}
        \caption{Azerty-based layout}
    \end{subfigure} 
    \begin{subfigure}[b]{0.3\textwidth}
        \includegraphics[width=\textwidth]{images/qwertz.png}
        \caption{Qwertz-based layout}
    \end{subfigure}
    \caption{Most common layouts in Europe}
\end{figure}
\clearpage

\section{Practice}
The more you know, the more you type. Once you are able to type a full word one needs to keep practising. This builds up a muscle memory of all the possible key stokes which make up the words of the dictionary.
The computerhope.com\citep{computerhope} website provide good everyday practice example to follow. There is also a large collection of games on the internet to practice and therefore improve the typing speed.\\
how-to-type.com \cite{howto} offers practice sessions such as 10 minutes to an hour per session as their recommendation.
The book written by \citet{handBook} also gives some advice regarding the finger position and the stretch to do beforehand.
There is no software which combines both, doing a check that the user has done finger stretches, and provide advice on how to position yourself in front of the keyboard improving your ability to type correctly. Each of these is done on an individual basis.

\section{Speed and accuracy}
Accuracy is the first skill we need to develop (typing for beginners)\cite{beginners}.\\
Firstly, the user must learn how to type all the letters of the keyboard, as said above. The user must know the exact position of each letter on the keyboard, without having to look at it. Once they know this, they will have the ability to type much faster. However faster can also mean more mistakes. If the mistake is found on the same letter, the user must practice typing this letter again and again to improve accuracy. The problem can also come from a finger which is not reactive enough. In this case, steps need to be put in place in order to train that particular finger. This will be one of the goals of the software to be created. Also, the user must be aware that the more improvement regarding typing speed, the harder it will become to improve this speed.(lingholic.com) \cite{plateau}. Below are two graphs explaining the difference between the expected learning curve and the actual learning curve.

\begin{figure}[H]
    \centering
    \begin{subfigure}[b]{0.4\textwidth}
        \includegraphics[width=\textwidth]{images/expected.jpg}
        \caption{Expected learning curve}
    \end{subfigure}
    \begin{subfigure}[b]{0.4\textwidth}
        \includegraphics[width=\textwidth]{images/actual.jpg}
        \caption{Actual learning curve}
    \end{subfigure}
    \caption{Expected and actual learning curve}
\end{figure}

\clearpage
\chapter{Conclusion}
From the current knowledge in typing, it appears that it is possible to quickly easily to learn to type faster and with a good accuracy. It is important for the user to learn the basics of the keyboard and the position of the letters. Once this is known, with the time and practice, typing speed will slowly and steadily increase.\\
Whenever the user has some accuracy problem with a single finger, practice will increase general precision.\\
For the general public, it they have a computer and a keyboard, they can learn how to type faster in three steps :
\begin{itemize}
	\item Learn the position of the letters on the keyboard and associate each fingers to each key
	\item Practice in order to improve the speed
	\item Correct the accuracy problems
\end{itemize}

\clearpage

%Bibliography
\begin{flushleft}
	\bibliographystyle{unsrtnat}
	\bibliography{lit_review.bib}
\end{flushleft}

\clearpage
\afterpage{\blankpage}

\end{document}
\chapter*{Keywords}
\begin{sortedlist}
  \sortitem{Keyboard}
  \sortitem{Keyboard layout}
  \sortitem{Fingers}
  \sortitem{Keys}
  \sortitem{Speed}
  \sortitem{Accuracy}
  \sortitem{Typing}
  \sortitem{Improvements}
  \sortitem{Learning}
\end{sortedlist}

\thispagestyle{empty}
\clearpage

\setcounter{page}{1}
\chapter{Abstract}
This review explores the way of improving the typing speed on the common keyboard.
Today's world is already full of keyboards, and it will never cease to increase. To be more efficient when using a computer, it is a good idea to learn how to type. On the web there are plenty of sources to improve your skills, but there is very little of them who really teach you how to type faster from scratch. The same goes for the existing software. And the main issue regarding this is that the software is not free.\\
The work presented here is based on books and websites that give advice on how to type speedily.\\
In order to address that issue, this review looks at the finger positions. Then it goes on to discuss how the user must practice to improve his or her typing skills. Lastly, it looks at the importance of accuracy and how this outweighs speed.

\clearpage
\chapter{Literature review}
\section{Some history}
In 1890, Lovisa Ellen Barnes wrote a book called "How To Become Expert in Typewriting"\cite{ref2} who was explaining how type on a Remington Typewriter.\\
In 1915, Dr. Henry Faulds wrote a book called "A Manual of Practical Dactylography"\cite{ref3}\\
An other example of book talking about dactylography.\cite{ref5}\\   
A more recent one published in 2012 and wrote by Faulds H.\cite{ref4}\\
As we can see, from a long time people provide way to learn how type so we can thereby say that way to type, and so typing speed, are always being important and are never been neglect by people.\\

\section{Keyboard}
The original layouts for the first mechanical typewriters were in alphabetical order (ABCDE etc..). Then different type of keyboard appeared according to languages used in each country (the most used is the QWERTY one but you can find AZERTY in France or QWERTZ in Germany for example).\\
Each of this layouts keyboard are classics one, that is mean when you buy a keyboard you will have one of this, but, since the invention of computers and indeed keyboards, a lots of research 	conducted to the invention of new non-classics layouts. Indeed, this new layouts had to made typing more easy and more comfortable. The most famous are Dvorak, Colemak and Workman.\\
Dvorak has been invented by August Dvorak and is the best alternative to QWERTY. It has been designed to increase typing speed, decrease errors and increase comfort.\\
After Dvorak, Colemak is the most popular. Also based on the QWERTY keyboard, it has been created in 2006 by Shai Coleman. It allow to promote the bearing of fingers on the middle rows of keys, it is better to maximise alternation between use right and left hands and minimize the movement of hands on the keyboard.\\
The third one is Workman. Thise one is less used than Dvorak and Colemak, but according with different developers, it is actually the best one to reduce fingers exhaustion. Indeed, it reduce usage of the two middle columns, reduce horizontal,diagonal and vertical finger stretching compared to Dvorak and Colemak.\cite{ref6}\\
To conclude about that, we can obviously say that from people have always trying to increase their typing skills by inventing new keyboards layouts which increase typing comfort and thus typing speed.     
\begin{figure}[h!]
   \begin{minipage}[b]{0.32\linewidth}
      \centering \includegraphics[scale=0.16]{images/KB_United_States_Dvorak.png}
      \caption{\it Dvorak Keyboard}
   \end{minipage}
   \begin{minipage}[b]{0.32\linewidth}   
      \centering \includegraphics[scale=0.16]{images/KB_US-Colemak_with_AltGr.png}
      \caption{\it Colemak Keyboard}
   \end{minipage}
   \begin{minipage}[b]{0.32\linewidth}
      \centering \includegraphics[scale=0.16]{images/KB_English_Workman.png}
      \caption{\it Workman Keyboard}
   \end{minipage}
\end{figure}
\clearpage

\section{Typing way and Fingers position}
To improve typing speed, in addition to change our keyboard, the only way is to practice. Actually, the average typing speed, calculate in Word Per Minute (WPM), is of 44 words.\cite{ref7} In the English language, the fastest typist in the world was Stella Pajunas who had a top speed of 216 words per minute. She realize it on a typewriter in 1946. Nowadays, it is easy with practice to had a typing speed of about 100 words par minute.\\ 
The typing way, which can be learn and improve on a lots of software and websites, is really the most important things to take in consideration to improve our typing speed. This typing way include the position of hands on a keyboard, which will change according to the type of keyboard, but for a QWERTY one it is like the figure 4 shows.
\begin{figure}[H]
\begin{center}
\includegraphics[width=9cm]{images/FingerHandPosUSA.png} 
\end{center}
\caption{\it Fingers position on QWERTY keyboard}
\label{Poulpy est multicolore}
\end{figure}
Here you can find some example of typing test on the internet:
\begin{itemize}
\item\it TypingTest.com\cite{ref8}
\item\it 10fastfingers.com\cite{ref9}
\end{itemize}
It also possible to find typing lessons on the internet.\\
To conclude about this part we can say that way to type and typing speed are things very democratize on the internet especially with a lots of website who can explain you best way to type, position of hands, give you advice and way to improve your typing style with speed test. That is showing that the development of a typing test is something useful.
Every book and website available regarding improving the typing speed are first mentions the finger positions.\\
As the book written by \cite{beginners}, the fingers must be positioned on the \textit{baseline}. The baseline, is the line on the keyboard containing the letters \textit{F} and \textit{J}. As with 'touch typing in ten lessons' \cite{tenlessons} book explains, every fingers should have an assigned letter. The figure below \ref{fing_pos} shows where the finger position is assigned on a QWERTY keyboard.
\begin{figure}[H]
	\centering
	\includegraphics[width=0.7\textwidth]{images/fing_pos.png}
	\caption{Position of the fingers on a QWERTY keyboard}\label{fing_pos}
\end{figure}
However, the layout of the keyboard may vary but the finger position will stay on the same baseline. www.computerhope.com \cite{computerhope} shows the different keys for each finger depending on the keyboard layout.
Other keyboards exist such as those represented in 'typing for beginners' \cite{beginners}. Unfortunately there are not common enough and are only used by more professional typists. So this literature review will not cover the use of them.\\
The user must know which finger is assigned to its specific letter on the keyboard. And so forth for each row. The book written by
\cite{beginners} explains the differents steps on how to follow the assignment for each row. Once again, the letters may vary depending on the keyboard layout, but the finger movements are still the same.
\begin{figure}[H]
    \centering
    \begin{subfigure}[b]{0.3\textwidth}
        \includegraphics[width=\textwidth]{images/qwerty.png}
        \caption{Qwerty-based layout}
    \end{subfigure}
	 \begin{subfigure}[b]{0.3\textwidth}
        \includegraphics[width=\textwidth]{images/azerty.png}
        \caption{Azerty-based layout}
    \end{subfigure} 
    \begin{subfigure}[b]{0.3\textwidth}
        \includegraphics[width=\textwidth]{images/qwertz.png}
        \caption{Qwertz-based layout}
    \end{subfigure}
    \caption{Most common layouts in Europe}
\end{figure}
\clearpage

\section{Practice}
The more you know, the more you type. Once you are able to type a full word one needs to keep practising. This builds up a muscle memory of all the possible key stokes which make up the words of the dictionary.
The computerhope.com\cite{computerhope} website provide good everyday practice example to follow. There is also a large collection of games on the internet to practice and therefore improve the typing speed.\\
how-to-type.com \cite{howto} offers practice sessions such as 10 minutes to an hour per session as their recommendation.
The book written by \cite{handBook} also gives some advice regarding the finger position and the stretch to do beforehand.
There is no software which combines both, doing a check that the user has done finger stretches, and provide advice on how to position yourself in front of the keyboard improving your ability to type correctly. Each of these is done on an individual basis.

\section{Speed and accuracy}
Accuracy is the first skill we need to develop (typing for beginners)\cite{beginners}.\\
Firstly, the user must learn how to type all the letters of the keyboard, as said above. The user must know the exact position of each letter on the keyboard, without having to look at it. Once they know this, they will have the ability to type much faster. However faster can also mean more mistakes. If the mistake is found on the same letter, the user must practice typing this letter again and again to improve accuracy. The problem can also come from a finger which is not reactive enough. In this case, steps need to be put in place in order to train that particular finger. This will be one of the goals of the software to be created. Also, the user must be aware that the more improvement regarding typing speed, the harder it will become to improve this speed.(lingholic.com) \cite{plateau}. Below are two graphs explaining the difference between the expected learning curve and the actual learning curve.

\begin{figure}[H]
    \centering
    \begin{subfigure}[b]{0.4\textwidth}
        \includegraphics[width=\textwidth]{images/expected.jpg}
        \caption{Expected learning curve}
    \end{subfigure}
    \begin{subfigure}[b]{0.4\textwidth}
        \includegraphics[width=\textwidth]{images/actual.jpg}
        \caption{Actual learning curve}
    \end{subfigure}
    \caption{Expected and actual learning curve}
\end{figure}

\clearpage
\chapter{Conclusion}
From the current knowledge in typing, it appears that it is possible to quickly easily to learn to type faster and with a good accuracy. It is important for the user to learn the basics of the keyboard and the position of the letters. Once this is known, with the time and practice, typing speed will slowly and steadily increase.\\
Whenever the user has some accuracy problem with a single finger, practice will increase general precision.\\
For the general public, it they have a computer and a keyboard, they can learn how to type faster in three steps :
\begin{itemize}
	\item Learn the position of the letters on the keyboard and associate each fingers to each key
	\item Practice in order to improve the speed
	\item Correct the accuracy problems
\end{itemize}

\clearpage

%Bibliography
\begin{flushleft}
	\bibliographystyle{plain}
	\bibliography{lit_review}
\end{flushleft}

\part{System analysis}

\chapter{Functional requirements}
As C++ need to be compile, we can provide to users an executable which is the only thing they need to run the software on their computer.

\section{What the software does}
This application is a learning and testing software using Qt framework and develop with C++ language. The software provide a way for users to learn from the beginning how type fast and well on a keyboard.
\\
The learning part is divided in forty-eight sections ("Learn" part in the GUI), each one gives to the user a training for different couple of letter. At the beginning when the user account is new, only the first training is available. When a user select a training couple, a keyboard layout is display (related to the layout the user select in his parameters) explaining where the letters are on the keyboard and which finger use for both. The first couple is "fj". When the training for the couple is completed, the training for the next couple is unlocked. The second couple is "dk", so when the user will start this training, he will start to type on 'd' or 'k' letter on the keyboard. However as the user type, letters from the first training will appears and user will now have to type on 'd','k','f' or 'j'. This process is the same for all along of the forty-eight learning sections, each time the user will have to manage the new couple of letter as well as the old ones.
\\
The practice part is divided in four sections ("Practice" part in the GUI), each one gives to the user an exercise. The first type of exercise is "Against time". As his name say, users has to type most words as possible in two minutes. At the end the software display some information, especially the number of word per minute.\\
The second type of exercise is "Normal", the software display a list of word, only in small letter, and the user has to type faster as possible and as before the application displays informations at the end.\\
The third type of exercise is "Improve". This one is a bit similar to learning exercise. Indeed it is based on the principal of repetition of some couple of letter that you have to type faster as possible.\\
The last type of exercise is "Text". This one is the most complete exercise because it regroups all you have learned before. Indeed, the text display contains small and capital letters, punctuation and word some words can be complex.
\\
The statistics part ("Statistics" part in the GUI) provides a way for users to get some statistics and is a good way for the user to get a rapid overview of his progression.
\\
The game part ("Game" part in the GUI) provides an interactive keyboard where user can see with which finger each letter on the keyboard may be type.
          

\section{Systems and subsystems}
The application contain one main system which is the software itself. As C++ is a language which need to be compile, the compilation provide a executable which is the only thing users need to run the software.
We do not need database to store any data, everything is stored in different files using DataStream.  

\section{Data requirements}
As said before, data are stored in different files using DataStream or a Qt class called QSettings. DataStrems is well support by Qt and is used to store informations in files. It is possible then to also read those files. QSettings is a class provide in the Qt library and can be used to store informations directly on the user desktop.

\subsection{Diagrams}
Next figures respectively shows the relationship diagram (figure \ref{Rhomepage}), the sequence diagram (figure \ref{Shomepage}) and the Use Case diagram (figure \ref{UChomepage}) (without details) for the home page. Basically, the home page works like this:
\begin{itemize}
\item The user start the software which is going to start the system and displaying the home page.
\item The user will have the choice between select and delete an existing user or create a new one. Those action are going to call different functions.
\end{itemize}

\begin{figure}[H]
	\centering
	\includegraphics[width=12cm]{diagrams/Rhomepage.png}
	\caption{Relationship diagram for Home page}
	\label{Rhomepage} 
\end{figure}
\begin{figure}[H]
	\centering
	\includegraphics[width=12cm]{diagrams/Shomepage.png}
	\caption{Sequence diagram for Home page}
	\label{Shomepage}
\end{figure}
\begin{figure}[H]
	\centering
	\includegraphics[width=12cm]{diagrams/UChomepage.png}
	\caption{User case diagram for Home page}
	\label{UChomepage}
\end{figure}

Next figures respectively shows the relationship diagram (figure \ref{Rlearnpage}), the sequence diagram (figure \ref{Slearnpage}) and the Use Case diagram (figure \ref{UClearnpage}) (without details) for the learn page. Basically, the learn page works like this:
\begin{itemize}
\item The user has already start the software and selected a user.
\item The system shows all learning exercise to the user (including already done and no done yet exercise).
\item The user the exercise he want and the system display the corresponding one in a new window.
\item When the exercise is done the system shows the results to the user.
\end{itemize}

\begin{figure}[H]
	\centering
	\includegraphics[width=12cm]{diagrams/Rlearnpage.png}
	\caption{Relationship diagram for Learn page}	
	\label{Rlearnpage}
\end{figure}
\begin{figure}[H]
	\centering
	\includegraphics[width=12cm]{diagrams/Slearnpage.png}
	\caption{Sequence diagram for Learn page}
	\label{Slearnpage}
\end{figure}
\begin{figure}[H]
	\centering
	\includegraphics[width=12cm]{diagrams/UClearnpage.png}
	\caption{User case diagram for Leran page}
	\label{UClearnpage}
\end{figure}

Next figures respectively shows the relationship diagram (figure\ref{Rpracticepage}), the sequence diagram (figure\ref{Spracticepage}) and the Use Case diagram (without details) for the practice page. Basically, the practice page works like this:
\begin{itemize}
\item The user has already start the software and selected a user.
\item The system shows all the four different exercise to the user.
\item The user select the one he want and the system will display it in a new window.
\item When the exercise is done the system shows the results to the user.
\end{itemize}

\begin{figure}[H]
	\centering
    \includegraphics[width=12cm]{diagrams/Rpracticepage.png}
    \caption{Relationship diagram for Practice page}  
    \label{Rpracticepage} 
\end{figure}
\begin{figure}[H]
	\centering
    \includegraphics[width=12cm]{diagrams/Spracticepage.png}
    \caption{Sequence diagram for Practice page}
    \label{Spracticepage}
\end{figure}
\begin{figure}[H]
	\centering
    \includegraphics[width=12cm]{diagrams/UCpracticepage.png}
    \caption{User case diagram for Practice page}
    \label{UCpracticepage}
\end{figure}

Next figures respectively shows the relationship diagram (figure\ref{Rgamepage}), the sequence diagram (figure\ref{Sgamepage}) and the Use Case diagram (without details) (figure\ref{UCgamepage}) for the game page. Basically, the game page works like this:
\begin{itemize}
\item The user has already start the software and selected a user.
\item Actually there is just one button in the game page so when the user click on it(or on a future one), the system is starting the appropriate game in a new window.
\end{itemize}

\begin{figure}[h]
	\centering
    \includegraphics[width=12cm]{diagrams/Rgamepage.png}
    \caption{Relationship diagram for Game page}
    \label{Rgamepage}   
\end{figure}
\begin{figure}[h]
	\centering
    \includegraphics[width=12cm]{diagrams/Sgamepage.png}
    \caption{Sequence diagram for Game page}
    \label{Sgamepage}
\end{figure}
\begin{figure}[h]
	\centering
    \includegraphics[width=12cm]{diagrams/UCgamepage.png}
    \caption{User case diagram for Game page}
    \label{UCgamepage}
\end{figure}

%------------------------------------
% Fin Partie Pierre
%------------------------------------
